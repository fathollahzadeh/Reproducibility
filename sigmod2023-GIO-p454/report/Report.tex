\documentclass[acmsmall,screen,pdfa]{acmart}

\def\BibTeX{{\rm B\kern-.05em{\sc i\kern-.025em b}\kern-.08em
    T\kern-.1667em\lower.7ex\hbox{E}\kern-.125emX}}

%% \BibTeX command to typeset BibTeX logo in the docs
\AtBeginDocument{%
  \providecommand\BibTeX{{%
    Bib\TeX}}}

\usepackage{pgfplots}
\usepackage{algorithm}
\usepackage[noend]{algorithmic}
\usepackage{listings}
\usetikzlibrary{arrows.meta, chains, positioning, shapes.symbols}
\usetikzlibrary{decorations,calligraphy}
\usepackage{numprint}
\usepackage{subfigure}
\usepackage{mathtools}
\usepackage{amsmath,mleftright}
\usepackage{multirow}
\newtheorem{theorem}{Theorem}
\newtheorem{definition}{Definition}
\newcommand{\eat}[1]{}
\usepackage{balance}


\newtheorem{definex}{Definition}
\npstyleenglish

\let\bbordermatrix\bordermatrix
\patchcmd{\bbordermatrix}{8.75}{4.75}{}{}
\patchcmd{\bbordermatrix}{\left(}{\left[}{}{}
\patchcmd{\bbordermatrix}{\right)}{\right]}{}{}

% math commands
\newcommand{\mat}[1]{\ensuremath{\mathbf{#1}}}
\newcommand{\card}[1]{\lvert #1\rvert}
\newcommand{\norm}[1]{\left\lVert#1\right\rVert}
\newcommand{\num}[1]{\numprint{#1}}

\DeclareMathOperator*{\argmin}{arg\,min}
\sloppy
\clubpenalty = 10000
\widowpenalty = 10000
\brokenpenalty = 10000
\frenchspacing

\AtBeginDocument{%
  \providecommand\BibTeX{{%
    \normalfont B\kern-0.5em{\scshape i\kern-0.25em b}\kern-0.8em\TeX}}}

\copyrightyear{2023}


% \received{October 2022}
% \received[revised]{January 2023}
% \received[accepted]{February 2023}

\setcopyright{acmlicensed}
\acmJournal{PACMMOD}
\acmYear{2023} \acmVolume{1} \acmNumber{2} \acmArticle{120} \acmMonth{6} \acmPrice{15.00}\acmDOI{10.1145/3589265}

%\keywords{Raw Data; Custom Data Format; Efficient Readers; Data Loading}

% \ccsdesc[500]{Information systems~Database management system engines}
% \ccsdesc[500]{Information systems~Data scans}
% \ccsdesc[500]{Information systems~Record and block layout}
% \ccsdesc[500]{Theory of computation~Database query processing and optimization (theory)}


\definecolor{codegreen}{rgb}{0,0.6,0}
\definecolor{codegray}{rgb}{0.5,0.5,0.5}
\definecolor{codepurple}{rgb}{0.58,0,0.82}
\definecolor{backcolour}{rgb}{0.95,0.95,0.92}


\lstset{
  language=bash,
  basicstyle=\ttfamily,
  backgroundcolor=\color{backcolour},  
  commentstyle=\color{codegreen},
  keywordstyle=\color{magenta},
  numberstyle=\tiny\color{codegray},
  stringstyle=\color{codepurple},
  basicstyle=\ttfamily\footnotesize,
  breakatwhitespace=false,         
  breaklines=true,                 
  captionpos=b,                    
  keepspaces=true,                 
  numbers=left,                    
  numbersep=5pt,                  
  showspaces=false,                
  showstringspaces=false,
  showtabs=false,                  
  tabsize=2
}

\lstdefinestyle{ascii-tree}{
  literate={├}{|}1 {─}{--}1 {└}{+}1 
}



\begin{document}
%\fancyhead{}

\title[GIO: Generating Efficient Matrix and Frame Readers for Custom Data Formats by Example]{GIO: Generating Efficient Matrix and Frame Readers for Custom Data Formats by Example\cite{GIO}}

\author{Saeed Fathollahzadeh} 
\orcid{0000-0003-3723-6191}
\affiliation{
  \institution{Graz University of Technology \& Know-Center GmbH}
  \country{Austria}}

\author{Matthias Boehm} 
%\authornote{This work was partially done at Graz University of Technology, Austria.}
\orcid{0000-0003-1344-3663}
\affiliation{
  \institution{Technische Universität Berlin}
  \country{Germany}}

\renewcommand{\shortauthors}{Saeed Fathollahzadeh \& Matthias Boehm}

\maketitle

This guide provides instructions for replicating the paper's findings. To ensure successful execution, it is essential to meet specific setup requirements.

\section{Source Code Info}
\begin{itemize}
  \item \textbf{Repository:} Apache SystemDS (\url{https://github.com/apache/systemds},\\ commit: \hyperlink{https://github.com/apache/systemds/commit/82d9d130861be8e36d37a08c22cdd8d3231de6c2}{82d9d130861be8e36d37a08c22cdd8d3231de6c2})
  \item \textbf{Reproducibility Repository:} \url{https://github.com/damslab/reproducibility/tree/master/sigmod2023-GIO-p454}
  \item \textbf{Programming Language:} \texttt{Java}, \texttt{Clang++10}, \texttt{Python 3.8}, \texttt{SystemDS} 
  \item \textbf{Packages/Libraries Needed:} \texttt{JDK 11}, \texttt{cmake}, \texttt{RapidJSON}, \texttt{Clang++}, \texttt{Python}, \texttt{Git}, \texttt{Maven}, \texttt{pdflatex}, \texttt{unzip}, \texttt{unrar}, \texttt{xz-utils}
\end{itemize}  

\section{Experiment Index Structure}
\begin{lstlisting}[style=ascii-tree]
  .
  ├── baselines
      ├── JavaBaselines
      ├── PythonPandas
      └── RapidJSONCPP
  ├── datagen
  ├── explocal
      ├── exp1_micor_bench_identification
      ├── exp2_identification_1k_10k
      ├── exp3_early
      ├── exp4_micro_bench
      ├── exp5_systematic
      ├── exp6_end_to_end
      └── plotting
  ├── load-had3.3-java11.sh
  ├── plots
  ├── run0LoadConfig.sh
  ├── run1SetupDependencies.sh
  ├── run2SetupBaseLines.sh
  ├── run3DownloadData.sh
  ├── run4GenerateData.sh
  ├── run5LocalExperiments.sh
  ├── run6PlotResults.sh
  └── runAll.sh
\end{lstlisting}


\section{Install and Config}
\textbf{Hardware and Software Info:} We ran all experiments on a server node an AMD EPYC 7302 CPU @ 3.0-3.3, GHz (16 physical/32 virtual cores) with 128 GB, two 480 GB SATA SSDs (system/home), and twelve 2 TB SATA HDDs. All reader experiments utilize a single SSD.

\textbf{Setup and Experiments:} The repository is pre-populated with the paper's experimental results (\texttt{"results"}), individual plots (\texttt{"plots"}), and SystemDS source code. 

To acquire installation and config, follow these steps:
\begin{lstlisting}
  ./run0LoadConfig.sh
  ./run1SetupDependencies.sh;
  ./run2SetupBaseLines.sh;
\end{lstlisting}

\textbf{Note 1.} \texttt{"run0LoadConfig.sh"} is crucial to execute before proceeding with other steps. It loads JVM memory settings and configures Hadoop settings. Please refrain from commenting out this line when running the steps 

\section{Data Preparation}
The datasets will be downloaded and generated automatically, follow these steps:
\begin{lstlisting}
  ./run3DownloadData.sh;
  ./run4GenerateData.sh;
\end{lstlisting}

\section{Run Experiments}
To execute all experiments, simply run \texttt{"run5LocalExperiments.sh"}. The experiments will commence, collecting output results.
\begin{lstlisting}
  ./run5LocalExperiments.sh;
\end{lstlisting}

\textbf{Note 2.} Our configuration entails running all experiments five times, making it a more time-consuming process.

\section{Plotting}
For plotting, simply run \texttt{"run6PlotResults.sh"}: 
\begin{lstlisting}
  ./run6PlotResults.sh;
\end{lstlisting}

Since the experiments are run five times, at the end of the experiments, we will merge all of them and create average results.


\section{Run All}
The entire experimental evaluation can be run via \texttt{"runAll.sh"}, which deletes the results and plots and performs setup, dataset download, dataset preparation, dataset generating, local experiments, and plotting. However, for a more controlled evaluation, we recommend running the individual steps separately.

\begin{lstlisting}
  ./run0LoadConfig.sh
  ./run1SetupDependencies.sh;
  ./run2SetupBaseLines.sh;
  ./run3DownloadData.sh;
  ./run4GenerateData.sh;
  ./run5LocalExperiments.sh;
  ./run6PlotResults.sh;
\end{lstlisting}

Please be aware that this process may take a considerable amount of time, and you may prefer to have better control over the timing of each experiment. 

\textbf{Note 3.} we are clearing the system cache using the command \texttt{"echo 3 >/proc/sys/vm/drop\_caches \&\& sync"}. Therefore, it is necessary to use \texttt{"sudo"} when executing \texttt{"runAll.sh"} (e.g., \texttt{"sudo runAll.sh"}).



\balance
\bibliographystyle{ACM-Reference-Format}
\bibliography{References}

\end{document}
\endinput

